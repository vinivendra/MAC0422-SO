\documentclass[11pt]{article}
\usepackage[brazilian]{babel}
\usepackage[utf8]{inputenc} %Deixa eu colocar letras com ascentos
\usepackage[T1]{fontenc}
\usepackage{amsmath}
\usepackage{color}
\usepackage{graphicx}


\title{Relatório - EP 3 \\ Sistemas Operacionais}



\begin{document}
    \maketitle

    \section{Integrantes}

    \begin{itemize}
        \item Victor Sanches Portella - Nº USP: 7991152
        \item Vinícius Jorge Vendramini - Nº USP: 7991103
    \end{itemize}


    \section*{Parte 1 - Compactação de memória}

    A parte 1 foi, em sua maioria, feita no arquivo alloc.c, na função compacta\_ep(). Ela é chamada quando o sistema fica sem memória, o que pode ser detectado nas funções de alocação ou por first fit (levemente alterada mas nativa do minix) ou por best fit (implementada por nós na outra parte).
    O funcionamento dela funciona assim (em linhas gerais): percorremos a lista de buracos. Para cada buraco, vemos na lista de processos se existe algum processo imediatamente após esse buraco. Se acharmos um processo, atualizamos o buraco (movendo ele para cima e o juntando com outros buracos quando necessário) e copiamos o processo para baixo. Caso tenhamos copiado o texto do processo, atualizamos outros processos que tiverem o mesmo texto para que apontem para o lugar certo; caso tenhamos copiado os dados, basta apenas atualizar as entradas relativas ao processo movido.

    Uma particularidade do Minix é a existência de um bloco no meio da memória que não corresponde a nenhum processo e que não é facilmente movido. Nós optamos por apenas pular esse bloco quando o acharmos e continuar a compactação depois dele. Além disso, nos nossos casos de teste, sempre havia o programa de teste em si rodando, o que criava um outro buraco quando ele parasse de rodar. Fora isso, a compactação funcionava para toda a memória.

    \section*{Parte 2 - Política de alocação}

    A implementação da segunda parte está mais espalhada pelo código. Ela usa uma variável global declarada no arquivo glo.h do pm para saber se deve usar um tipo ou outro de alocação. Essa variável então é checada a cada pedido de alocação, para saber se devemos chamar a função de first\_fit ou de best\_fit. A chamada de sistema criada, com nome \textit{ep\_uses\_best\_fit(type)}, recebe 0 se não deve usar best fit (e portanto deve usar first fit) ou 1 se deve usar best fit.

    \section*{Parte 3 - Mapa de memória}

    A terceira parte está quase toda no dmp\_pm.c. Ela envolve uma função que copia e ordena a tabela de processos, pega as informações restantes necessárias através de uma chamada getsysinfo e então imprime as informações do processo necessárias, seguidas das informações de memória. As implementações referentes à chamada getsysinfo e aos pedidos de informações de memória estão principalmente no alloc.c e no misc.c.


    \subsection*{Observações}
    Apenas para fins de testes, criamos uma outra condição na chamada de sistema (ep\_uses\_best\_fit). Como ela usava apenas valores 0 ou 1, pudemos usar quaisquer outros valores para executar uma chamada excepcional à função de compactação de memória. Assim, para testar essa função, basta chamar (por exemplo) ep\_uses\_best\_fit(3).

    Os arquivos de testes estão na pasta /usr/src/EP3-Testes. Um deles contém uma chamada à ep\_uses\_best\_fit para compactar a memória; o outro chama a ep\_uses\_best\_fit com o valor passado para alternar entre first\_fit e best\_fit.

\end{document}
