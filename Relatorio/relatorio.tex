\documentclass[11pt]{article}
\usepackage[brazilian]{babel}
\usepackage[utf8]{inputenc} %Deixa eu colocar letras com ascentos
\usepackage[T1]{fontenc}
\usepackage{amsmath}
\usepackage{color}
\usepackage{graphicx}


\title{Relatório - EP Fase 4 \\ Laboratório de Programação 2}



\begin{document}

\maketitle

\section{Integrantes}

\begin{itemize}

\item Victor Sanches Portella - Nº USP: 7991152

\item Vinícius Jorge Vendramini - Nº USP: 7991103



\end{itemize}

\section{Parte 1}

Para a primeira parte do EP criamos uma função \textbf{dmp\_ep} no arquivo \textit{dmp\_kernel.c}, onde ficava inicialmente a implementação da função chamada pelo \textbf{F4}. O funcionamento da função é simples e similar às outras funções de dump. São criados um vetor de processos e alguns contadores (estáticos, para possibilitar a paginação). Esse vetor é então ordenado pela prioridade e serve de fonte de dados para as impressões seguintes. As informações são tiradas então tanto do próprio vetor (vindo da tabela proc) quanto de uma outra tabela (a mproc) e impressas de acordo.


\section{Parte 2}

Para fazer a parte 2, criamos uma System Call chamada \textbf(setpriority\_ep(pid, 
pri)), onde \textbf{pid} e \textbf{pri} são inteiros representando respectivamente
o pid do processo alvo e a nova prioridade desse processo.

Essa System Call cria uma mensagem, que é enviada para o Process Manager, que chama
a função \textbf{do\_setpriority\_ep()}. Lá verificamos se o processo chamador é 
pai do processo alvo, além de testar se o PID passado é de fato válido e se a prioridade dada como argumento é prioridade de usuário. Caso passe nos dois testes, chamamos a Kernel Call \textbf{sys\_nice}, que enviará uma mensagem para o System Task. Lá será verificado se a prioridade passada é válida,

Caso não haja nenhum problema, a prioridade é mudada na tabela de processos do
Kernel, e o processo é mudado de fila de prioridade.

Valores de retorno possíveis da system call:\footnote{Não seguimos o padrão do enunciado pois tivemos problemas ao tentar fazer a função retornar valor negativo, além de que ficaria ambíguo em determinados casos.}

\begin{itemize}

	\item \textbf{Retorno 0:} Tudo ocorreu bem.

	\item \textbf{Retorno -1:} Prioridade inválida.

	\item \textbf{Retorno 1:} PID inválido ou o processo alvo não pode ter a prioridade alterada.

	\item \textbf{Retorno 2:} Processo alvo não é filho do processo chamador.

\end{itemize}

É importante notar que a prioridade \textbf{pri} passada é um número entre -20 e 20, 
não representando a real fila para a qual o processo será re-alocado. Quanto mais alto 
for \textbf{pri}, maior será a prioridade do processo. Dado esse número, o \textbf{sys\_nice} faz uma conta para ter uma equivalência com relação ao número da fila
que o processo deveria ir. Aqui fizemos uma tabela mostrando e qual fila o processo será
realocado para cada valor de \textbf{pri}.


\section{Teste}

Nossa imagem do MINIX tem senha no root. Para acessar, use:

\begin{itemize}
	\item \textbf{Login:} root
	\item \textbf{Senha:} senha
\end{itemize}

No endereço \textbf{/usr/src/EP2-testes} temos os arquivos de teste para verificar o funcionamento de nossa chamada ao sistema. 



\end{document}