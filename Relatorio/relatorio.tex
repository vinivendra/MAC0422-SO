\documentclass[11pt]{article}
\usepackage[brazilian]{babel}
\usepackage[utf8]{inputenc} %Deixa eu colocar letras com ascentos
\usepackage[T1]{fontenc}
\usepackage{amsmath}
\usepackage{color}
\usepackage{graphicx}


\title{Relatório - EP Fase 4 \\ Laboratório de Programação 2}



\begin{document}

\maketitle

\section{Integrantes}

\begin{itemize}

\item Victor Sanches Portella - Nº USP: 7991152

\item Vinícius Vendramini - Nº USP: XXXXXXX



\end{itemize}

\section{Parte 1}

!!!!


\section{Parte 2}

Para fazer a parte 2, criamos uma System Call chamada \textbf(setpriority\_ep(pid, 
pri)), onde \textbf{pid} e textbf{pri} são inteiros representando respectivamente
o pid do processo alvo e a nova prioridade desse processo.

Essa System Call cria uma mensagem, que é enviada para o Process Manager, que chama
a função \textbf{do\_setpriority\_ep()}. Lá verificamos se o processo chamador é 
pai do processo alvo, além de testar se o PID passado é de fato válido. Caso 
passe nos dois testes, chamamos a Kernel Call \textbf{sys\_nice}, que enviará uma
mensagem para o System Task. Lá será verificado se a prioridade passada é válida,
!!!!!!além de alguns outros testes de segurança!!!!!!.

Caso não haja nenhum problema, a prioridade é mudada na table de processos do
Kernel, e o processo é mudado de fila de prioridade.

É importante notar que a prioridade \textbf{pri} passada é um número entre -20 e 20, 
não representando a real fila para a qual o processo será re-alocado. Quanto mais alto 
for \textbf{pri}, maior será a prioridade do processo. Dado esse número, o \textbf{sys\_nice} faz uma conta para ter uma equivalência com relação ao número da fila
que o processo deveria ir. Aqui fizemos uma tabela mostrando e qual fila o processo será
realocado para cada valor de \textbf{pri}.

!!!!Falta tabela!!!!!

\section{Teste}



\end{document}